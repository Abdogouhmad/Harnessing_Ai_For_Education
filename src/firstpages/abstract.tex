
\newenvironment{myabstract}
{%\clearpage           % we want a new page          %% I commented this
    \thispagestyle{empty}% no header and footer
    \vspace*{\stretch{2}}% some space at the top
    %\itshape       % the text is in italics
    % \centering
    \justifying      % flush to the right margin
}
{\par % end the paragraph
    \vspace{\stretch{3}} % space at bottom is three times that at the top
    \clearpage           % finish off the page
}

\chapter*{}
\addcontentsline{toc}{chapter}{\textbf{ABSTRACT}}
\begin{myabstract}
    \begin{center}
        {\textbf{ABSTRACT}}
    \end{center}
    This research paper explores using artificial intelligence (AI) to
    enhance learning experiences in the humanities, focusing on English
    university students at Hassan II University in Casablanca, Morocco.
    The study investigates students’ perceptions and experiences with
    AI-driven tools to determine if these technologies can improve
    academic engagement and performance. It also examines the
    challenges and opportunities associated with integrating
    AI into higher education in Morocco, specifically within
    the humanities department. A mixed-methods approach was
    employed, combining quantitative surveys and qualitative
    analysis. Surveys were administered to gather data on students’
    perceptions and experiences with AI-driven tools. Participants
    included 60 English department students from different academic
    years. Data collection was carried out via WhatsApp, ensuring
    diverse and comprehensive responses. Through a thorough analysis
    of data and previous, the findings demonstrate the positive impact
    of AI on learning outcomes and emphasize the importance of embracing
    AI technologies in educational settings. This research provides
    valuable insights for educators, policymakers, and researchers
    interested in using AI to enhance learning experiences and
    promote academic success in the digital age.
\end{myabstract}
