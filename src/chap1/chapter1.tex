\chapter{INTRODUCTORY CHAPTER}\label{ch:introductory-chapter}
%\addcontentsline{toc}{chapter}{INTRODUCTORY CHAPTER}
\section{Problem statement}\label{sec:problem-statement}
\justifying
Artificial intelligence has taken all over the industries and become a revolutionized technology.
It potentially transforms various industries to be more productive \citep{Dirk_Czarnitzki}.
However, the emergence of similar AI-driven tools like \say{ChatGPT}, which have significance
capabilities, there remains a massive gap comprehending how to effectively
interact with it,
especially that these tools have gained prominence across
sectors since their launch in late November 2020 \citep{MarrB_2023}; their full potential
has not yet been used within the realm of education to foster genuine
engagement and knowledge acquisition among humanities students.
This has arisen questions about practical ways of integrating these tools in this context.
Therefore, the key focus of this study lies in exploring how AI can be effectively integrated into education
to enhance learning experiences within the humanities.
The significance of this problem goes beyond implementing technology;
it involves transforming education practices and methodologies.
Using AI in higher education, especially in humanities, can potentially
revolutionize it.
Facilitating personalized learning, encouraging critical thinking skills, and
enhancing engagement during lectures \citep{baskara_personalised_2023}.
Addressing this gap is vital for improving the quality and effectiveness of humanities education ensuring
that students have the skills to succeed in an increasingly digital and interconnected society.
Therefore, exploring ways to use AI in education is an effort with significant implications,
for the future of learning and acquiring knowledge.
% 2nd section
\section{The purpose of the study}\label{sec:the-purpose-of-the-study}
\justifying
This study examines practical ways of integrating artificial intelligence (AI) into the humanities.
It also investigates effective AI-driven tools for improving learning experiences in higher education.
To better understand students’ perceptions and experiences, this study explores students’ attitudes toward their academic performance using AI-driven tools.
Furthermore, this study examines the challenges and opportunities associated with the use of AI in higher education, specifically in the humanities.
By addressing these objectives, this study aligns with the goal of enriching learning experiences in humanities disciplines.

% 3rd
\section{The Rationale and significance of the study}\label{sec:the-rationale-and-significance-of-the-study}
\justifying
The widespread accessibility and prevalence of AI shows that 73\% of US companies have already
implemented AI into some aspects of their businesses as \citep{pricewaterhousecoopers} reports.
Consequently, the fame of using AI in recent years prompted researchers to investigate practical
ways of using AI tools for enhancing human productivity across various fields,
including education.
This study delves into AI-driven tools within a framework aimed at addressing
how they can be effectively used to enhance learning experiences within humanities.

\section{Research questions and hypotheses}\label{sec:research-questions-and-hypotheses}
\subsection{Research questions}\label{subsec:research-questions}
\justifying
The study seeks to investigate the potential ways
of harnessing Artificial Intelligence for
Enhanced Learning Experiences in the Humanities.
Hence,
the following research questions will be addressed in this paper:
\begin{itemize}
	\item What are the most effective ways to use AI-driven
	      tools for enhancing learning experiences in higher education,
	      especially in the humanities?
	\item What are students’ attitudes toward their academic performance
	      while using AI-driven tools?
	\item What are the challenges and opportunities associated
	      with using AI in higher education in Morocco,
	      specifically in the humanities?
\end{itemize}
\subsection{Hypotheses}\label{subsec:hypotheses}
\justifying
\noindent
Following intended objectives, these hypotheses have been developed:
\begin{itemize}
	% \item Students who use AI-driven tools reveal better learning outcomes
	%       compared to those who do not in higher education, specifically in the humanities.
	\item AI-driven tools are singificantly fostering engagment in academic setting.
	\item AI-driven tools are significantly improving academic
	      performance and engagement in the humanities.
	\item There are challenges and opportunities are associated with using AI in higher
	      education in Morocco, specifically in the humanities.
\end{itemize}

% 5th


\section{The Organization of the paper}\label{sec:the-organization-of-the-paper}
\justifying
The monograph comprises five chapters, each serving a purpose within this study.
The first chapter gives an overview of the study discussing its problem,
purpose, rationale, significance, questions and hypotheses.
The second chapter review of relevant literature.
It reviews the most existing studies on AI in education to highlight current
trends, challenges, and potential strategies for using AI-driven tools.
This chapter explores emerging trends, challenges, and practical approaches for using AI-driven tools.
The third chapter is designed to provide a comprehensive explanation of data-collection.
It describes the research design, participants, instrument, and relevant procedures
adopted for analysis.
The finding chapter will analysis, interpret, and discuss data-collection in depth.
The chapter also aims to either validate or reject the hypotheses of the study.
Finally, the concluding chapter will focus on a summary of research objectives, methodology, and findings.
Furthermore,
this chapter will address the study's limitations and implications while offering suggestions for further studies.
