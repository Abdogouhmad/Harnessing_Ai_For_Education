\chapter{DISCUSSION \& ANALYSIS}
\section{Introduction}
After clarifying the data collection process and the relevant procedures adopted for
the analysis, this chapter aims to examine the previously collected data in order to determine
the definitive truth regarding hressing AI into higher education. To this end, statistical
tables and graphs would be disclosed. The findings of this
paper are consistent with previous research results on the practical ways of implementing AI into
humanities, students' point of view about their preformance while using AI, challanges and oppurtunites.
In short, the survey’s results indicate that students academic performance is improved while using AI.
\section{Results}
% In this sections, a descriptive statistic technique was used to
% interpret the data collected from English
% university students through filling out Google form.
The data collected from the survey conducted among English university students
revealed compelling insights into the students' perceptions and experiences with AI-driven tools.
The analysis utilized descriptive statistical techniques to interpret the responses gathered through the questionnaire.
The questionnair divided into two main parts: the first part is to investigate
whether the students' academic performance have improved or not while using AI-driven. The second
part is about challenges and oportunites faced during using AI for academic studies.

\subsection{The students' academic performance while using AI-driven tools}
\begin{figure}[h]
	\centering
	\includegraphics[width=11cm, height=7cm]{./chap4/figures/prf}
	\captionof{figure}{perceive academic performance using AI-driven tools}
\end{figure}
After assessing the respondents’ opinions on the use of AI-driven tools
and students' academic preformance, it is obvious that the majority
of their academic preformance was improved. it is clearly shown that
74.3\% agreed that AI improve their academic performance. In addition to
20\% of respondents believed that the use of AI didn't affect their academic
performance. Whereas 4.7\% showed their negative resulte believing that AI
affect their academic preformance to be declined while using it.

\subsection{The students' engagments while using AI-driven tools in academic setting}

\begin{figure}[H]
	\centering
	\includegraphics[width=11cm, height=7cm]{./chap4/figures/engagment}
	\captionof{figure}{perceive academic engagment using AI-driven tools}
\end{figure}

The data collected on the extent of students engagement when using AI-driven tools
provides a diverse range of percaptions and experiences. While a significant portion
of respondents reported moderate engagment 34.8\%, indicating a level of engagment with
these AI-driven tools, an almost equal of neuturality 34\%, suggesting a mixed reception.
However, a notable subset of students reported feeling highly engaged 20.8\%, depicting that
these tools can effectively encourage students to be engaged. Negatively, the presence
of respondents feeling not very engaged 7.5\% or not engaged at all \% highlights potential
limitations or challenges associated with the implementation or usage of AI-driven tools.


\subsection{Challenges faced while using AI-driven tools in academic setting}

\begin{figure}[H]
	\centering
	\includegraphics[width=17cm, height=9cm]{./chap4/figures/chall}
	\captionof{figure}{challenges faced when using AI-driven tools in academic setting}
\end{figure}

The bar chart shows the results of a survey on the challenges students
face when using AI-driven tools in their academic studies. Out of 44 respondents,
the biggest challenge was reported to be insufficient trainging for the
tools, with 45.5\% of students selecting this options. While resistance or skepticism
from professors towards using AI-driven tools was reported to be the second challenge
face by students with 40.9\%.
Following this was lack of access to reliable internet, with 31.8\% of students
reporting this as a challenge.
A smaller percentage of students reported facing challenges with pricacy and data security matters
with 25\%. while 2.3\% was bias, misinformation, propaganda in the AI tools and high prices of the AI-driven tools.

% The blow bar chart has shown that 45.5\% of English university
% students face Insufficient training
% as main challenge. In addition 40.9\% was resistance
% or skepticism from professors towards using AI-driven tools in
% academic settings. While 31.8\% and
% 25\% were lack of access to relaible internet and privacy comcerns. whereas
% 2.3\% was bais and high plans prices that AI-driven tools offer.
\subsection{AI-driven tools and opportunities for improving learning experiences in the humanities }
\begin{figure}[H]
	\centering
	\captionof{figure}{AI-driven tools and oppu}
	\includegraphics[width=11cm, height=7cm]{./chap4/figures/op.png}
\end{figure}

The majority of English university students reported with 50.9\% that AI-driven tools offer opportuinities
for enhancing learning experiences in the humanities. In addition to 28.9\% to report the
ignorance of the opportunities that AI-driven tools offer. Whereas 20.8\% negatively reported
the idea of AI-driven tools offers such opportunity for enhancing the learning experience
within humanities setting.


Those who answered with yes a question is asked to get some examples from students
to understand what kind of opportunities AI-driven tools can offer to enhance the learning experiences and answers were follow:

% 1st resp
\resp{1}{AI-driven tools in higher education could enhance personalized learning
	through adaptive learning platforms and virtual
	tutoring, and improve research with data analysis tools.}
% 2nd resp
\resp{2}{Research Assistance: AI can assist
	researchers in analyzing vast amounts of
	data, generating hypotheses, and identifying trends,
	accelerating the pace of discovery.}
% 3rd resp
\resp{3}{It can help students be more self-reliant
	and seek knowledge wherever and whenever they want to.}
%4rd resp 
\resp{4}{Making student more used to communicate with chat
	bot ai characters when they can't find native speakers to communicate with.}

As it is shown the majority of english students to claim that AI-driven tools can
help students to boost their productivity through analysing the data for researcheres
and provide personalized learning through adaptive learning platforms and virtual tutoring.

\section{Discussion}
It would be more fitting to restate the current study’s goals and research hypothesis
before digging deeper into the research findings and discussion. As mentioned in the
introduction, the main objective of this paper is to evaluate English students’ attitudes towards the
use of AI-driven tools for enhancing their academic prefromance and learning experiences.
learners. Accordingly, the recent study has significantly helped
to validate or reject the hypotheses stated in the previous chapter.
These later were formed as follows:
\begin{itemize}
	\item AI-driven tools are singificantly fostering engagment in academic setting.
	\item AI-driven tools are significantly improving academic performance in the humanities.
	\item There are challenges and opportunities are associated with using AI in higher education
	      in Morocco, specifically in the humanities
\end{itemize}
The current study’s findings strongly demonstrate positive attitudes by students towards
the use AI-driven tools as learning tool. The students have been asked precise questions
to explore their perceptions on the use of AI-driven tools

As data results reveal, the major part of participants believed that AI-driven tools
improved their academic performance and foster their engagment. In the same way, 
\citep{mohammed_exploring_2023} performed a related study
to demonstrate that AI-driven tools can effectively enhance 
academic preformance and foster engagment. Results showed
that the majority of respondents agreed that \say{ChatGPT} 
motivates and engages students by offering access to 
many resources and improving academic perormance. 
