\chapter{INTRODUCTORY CHAPTER}\label{ch:introductory-chapter}
%\addcontentsline{toc}{chapter}{INTRODUCTORY CHAPTER}


\section{Problem statement}\label{sec:problem-statement}
\justifying


Artificial intelligence has taken all over the industries and become a revolutionized technology.
It potentially transforms industries to be more productive \textcite{Dirk_Czarnitzki_artificial_2023}.
However, the emergence of similar AI-driven tools like \say{ChatGPT}, which have significance
capabilities, there remains a massive gap comprehending how to effective
interact with it. Especially, these tools have gained prominence across
sectors since their launch in late November 2020 \textcite{MarrB_2023} their full potential
has yet been utilized within the realm of education to foster genuine
engagement and knowledge acquisition among humanities students. This has
arisen questions about practical ways of integrating these tools in this context.
Therefore, the key focus of this study lies in exploring how AI can be harnessed in education
to enhance learning experiences within the humanities environment.


% Artificial intelligence is a revolutionary technology that can potentially transform any industry, including education, to be productive.
% Despite the emergence of AI and AI-driven tools like “ChatGPT,” which have significant capabilities,
% there remains a gap in comprehending how effectively AI can enhance learning experiences within humanities.
% regardless that “ChatGPT” and similar AI-driven tools have gained prominence across various industries since its release in late November 2022, they have
% not been fully utilized in education to promote genuine educational engagement and knowledge acquisition among humanities students.
% Consequently, questions have arisen about the efficacy of using these tools in higher education.
% Therefore, the central problem of this study is How AI can be harnessed in education to enhance learning experiences in the humanities.





% 2nd section


\section{The purpose of the study}\label{sec:the-purpose-of-the-study}
\justifying
The study focuses on the practical ways of implementing AI-driven tools to enhance humanities learning experiences.
Hence, The research paper will examine different ways that can be used to harness it in higher education
% 3rd


\section{The Rationale and significance of the study}\label{sec:the-rationale-and-significance-of-the-study}
\justifying
The widespread accessibility and prevalence of AI shows that 73\% of US companies have already
implemented Ai into some aspects of their businesses as \textcite{pricewaterhousecoopers} reports.
Consequently, the fame of using AI in recent years prompted researchers to investigate practical
ways of utilizing AI tools for enhancing human productivity across various fields,
including education. This study
delves into AI-driven tools within a framework aimed at addressing
how they can be effectively utilized to enhance learning experiences within humanities.


\section{Research questions and hypotheses}\label{sec:research-questions-and-hypotheses}

\subsection{Research questions}\label{subsec:research-questions}
\justifying
The study seeks to investigate the potential ways
of harnessing Artificial Intelligence for
Enhanced Learning Experiences in the Humanities.
Hence,
the following research questions will be addressed in this paper:
\begin{itemize}
      \item What are the most effective ways to utilize AI-driven
            tools for enhancing learning experiences in higher education,
            especially in the humanities?
      \item What is the impact of AI-driven pedagogical tools
            on university students' academic performance
            and engagement in the humanities?
      \item What are the challenges and opportunities associated
            with using AI in higher education in Morocco,
            specifically in the humanities?
\end{itemize}
\subsection{Hypotheses}\label{subsec:hypotheses}
\justifying
Following intended objectives, these hypotheses have been developed:
\begin{itemize}
      \item Students who use AI-driven tools reveal better learning outcomes
            compared to those who do not in higher education, specifically in the humanities.
      \item AI-driven tools are significantly improving academic
            performance and engagement in the humanities.
      \item There are challenges and opportunities are associated with using Ai in higher
            education in Morocco, specifically in the humanities.
\end{itemize}

% 5th


\section{The Organization of the paper}\label{sec:the-organization-of-the-paper}
\justifying
The monograph comprises five chapters, with each chapter contributing to the study.
Therefore, the study is structured as follows: The first chapter serves as an overview of the study.
It includes the study problem, the purpose of the study, its rationale and significance, and the study's questions and hypotheses.
The second chapter is a review of the literature.
It focuses on studies conducted on the use of AI in education to showcase current trends, challenges, and practical strategies for utilizing AI-driven tools.
The third chapter is made up to provide a comprehensive explanation of data collection.
It describes the research design, participants,  instrument, and relevant procedures adopted for analysis.
The findings chapter will analyze, interpret, and discuss data collection in depth.
Additionally, the chapter aims to validate or reject the study's hypotheses.
Lastly, in the concluding chapter, the attention will be directed towards summarizing research objectives, methodology, employed and critical findings.
Furthermore, this section will address the study's limitations and implications while providing suggestions for further studies.

% new txt
%The second chapter is a review of literature.
%It reviews the most existing studies on Ai in education to highlight current
%trends, challenges, and potential strategies for utilizing AI-driven tools.
%The third chapter is designed to provide a comprehensive explanation of data-collection.
%It describes the research design, participants, instrument, and relevant procedures
%adopted for analysis.
%The finding chapter will analysis, interpret, and discuss data-collection in depth.
%Additionally, the chapter aims to either validate or reject the hypotheses of the study.
%Finally, the concluding chapter will focus on a summary of research objectives,
%methodology, and findings.
%In addition, it will contain the shortcomings and implications of the study,
%as well as suggestions for further studies.


