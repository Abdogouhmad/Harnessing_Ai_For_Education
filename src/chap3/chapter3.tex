\chapter{METHODOLOGY}
\section{Introduction}
The present chapter aims to illustrate the research design employed
to investigate the integration of Artificial Intelligence in enhancing
learning experiences in the Humanities. It details the research design,
participants, instruments, data collection, and analysis procedures.
The chosen methodology will be justified for its relevance according to
research questions and hypotheses posted in Chapter 1, section \ref{sec:research-questions-and-hypotheses},
aiming to provide insight into the impact of AI-driven tools in higher education.

% The goal of the present chapter is to illustrate the research design
% employeed to to investigate the integration of Artificial Intelligence
% in enhancing learning experiences in the Humanities. it details
% the research design, participants, instruments, data collection, and
% analysis procedures. The chosen methodology will be justified for its
% relevance according to research quations adn hypothesis posted in chapter 1
% section: \ref{sec:research-questions-and-hypotheses}, aiming to provide insight
% over the impact of AI-driven tools in higher education.
\section{Objectives of the study}
This study's main objective is to examine the effectiveness, usefulness,
and benefits of using AI-driven tools as learning tools to enhance students’
academic performance. The purpose is to discover students’ perceptions and
the practical ways they use AI tools to enhance their academic performance.
Accordingly, the following research questions and hypotheses have been
formed based on the objectives.
% The main of the objective of this study is to examine the effectiveness, usefulness and
% benefits of using AI-driven tools as learning tool to enhance the students' academic preformance.
% The purpose tends to discover students' presception along with the effective ways they use AI tools
% to enhance their academic preformance. Accordingly to the objectives the following research questions
% and hypotheses have been formed.
\section{Research Questions and hypotheses}
\subsection{Research Questions}
The study aims to investigate hressing AI within higher education espacially humanities. Hence,
the following question are formulated:
\begin{itemize}
	\item What are the most effective ways to use AI-driven
	      tools for enhancing learning experiences in higher education,
	      especially in the humanities?
	\item What are students’ attitudes toward their academic performance
	      while using AI-driven tools?
	\item What are the challenges and opportunities associated
	      with using AI in higher education in Morocco,
	      specifically in the humanities?
\end{itemize}

\subsection{Hypothese}
In terms of achieving the current purpose, the follwing hypotheses have been formulated:
\begin{itemize}
	\item AI-driven tools are singificantly fostering engagment in academic setting.
	\item AI-driven tools are significantly improving academic
	      performance and engagement in the humanities.
	\item There are challenges and opportunities are associated with using AI in higher
	      education in Morocco, specifically in the humanities.
\end{itemize}

\section{Research design}
This research employs a mixed-methods approach to explore the efficacy of AI-driven
tools in enhancing learning experiences in the Humanities. Surveys will serve as the
primary data collection method to gather insights on the effective integration of AI-driven
tools into the educational process. Furthermore, the surveys will gather students' views on
their academic performance when using these tools. By combining quantitative survey data with
qualitative student perspectives, this study seeks to provide a comprehensive analysis of the
impact of AI technology on learning outcomes and student engagement in the Humanities department.

\section{Participants}
The current study investigates the attitudes of one major group, English university students
from different academic years: first year, second year, and third Year. The survey was
shared on WhatsApp. The Participants were asked relevant questions that contributed to the study.
They comprise 53 students from the English department (40 are female, 13 are male).
The age, gender, and previous experience with AI distribution of this paper’s
participants are shown in the table below.

\begin{table}[H]
	\captionof{table}{Participants by Age and Gender}
	\begin{tabular}{l|ccc|ccc|}
		\cline{2-7}
		\multirow{2}{*}{}                & \multicolumn{3}{c|}{Gender}  & \multicolumn{3}{c|}{Age}                                                                                                                                                \\ \cline{2-7}
		                                 & \multicolumn{1}{c|}{female}  & \multicolumn{1}{c|}{male}    & \multicolumn{1}{c|}{prefer not to say} & \multicolumn{1}{c|}{18-25}   & \multicolumn{1}{c|}{26-35}   & \multicolumn{1}{c|}{36 and above} \\ \hline
		\multicolumn{1}{|l|}{Frequancy}  & \multicolumn{1}{c|}{40}      & \multicolumn{1}{c|}{13}      & \multicolumn{1}{c|}{NULL}              & \multicolumn{1}{c|}{33}      & \multicolumn{1}{c|}{11}      & \multicolumn{1}{c|}{9}            \\ \hline
		\multicolumn{1}{|l|}{percentage} & \multicolumn{1}{c|}{75.5 \%} & \multicolumn{1}{c|}{24.5 \%} & \multicolumn{1}{c|}{NULL}              & \multicolumn{1}{c|}{62.3 \%} & \multicolumn{1}{c|}{20.8 \%} & \multicolumn{1}{c|}{17 \%}        \\ \hline
	\end{tabular}
\end{table}
% another table is here
\begin{table}[H]
	\begin{tabular}{l|ccc|cc|}
		\cline{2-6}
		\multirow{2}{*}{}                & \multicolumn{3}{c|}{Academic year} & \multicolumn{2}{c|}{The usage of AI for academic purposes}                                                                                             \\ \cline{2-6}
		                                 & \multicolumn{1}{c|}{First Year}    & \multicolumn{1}{c|}{Second Year}                         & \multicolumn{1}{c|}{Third Year} & \multicolumn{1}{c|}{Yes}   & \multicolumn{1}{c|}{No}    \\ \hline
		\multicolumn{1}{|l|}{Frequancy}  & \multicolumn{1}{c|}{5}             & \multicolumn{1}{c|}{16}                                  & \multicolumn{1}{c|}{31}         & \multicolumn{1}{c|}{44}    & \multicolumn{1}{c|}{9}     \\ \hline
		\multicolumn{1}{|l|}{percentage} & \multicolumn{1}{c|}{9.4 \%}        & \multicolumn{1}{c|}{30.2 \%}                             & \multicolumn{1}{c|}{58.5 \%}    & \multicolumn{1}{c|}{83 \%} & \multicolumn{1}{c|}{17 \%} \\ \hline
	\end{tabular}
	\captionof{table}{Participants by Academic Year and usage of AI for academic purposes}
\end{table}


\section{Instrument}
\justifying

Various data collection instruments and tools are used by research to collect, measure, and interpret data related to this study.
To that end, one of these instruments was used, namely \say{the questionnaire.} The questionnaire comprises of two types
of questions: factual and attitudinal. The factual questions are aimed at identifying certain demographic characteristics
of the participants, such as their gender, age, academic year, and department. Whereas, the attitudinal questions
are intended to investigate the students' attitudes towards utilizing AI-driven tools in higher education,
especially in the humanities department.


The questionnaire design primarily employs a closed-ended format, which pre-determines response
options for the participants. However, it also incorporates three open-ended questions to elicit
detailed responses and qualitative insights from the participants. These open-ended questions are
designed to encourage the participants to provide examples of opportunities for integrating AI-driven
tools in higher education, offer suggestions for enhancing the integration of AI in higher education,
share their experiences of using AI-driven tools, and provide feedback or comments regarding the AI-driven initiatives.


% To collect, measure, and interpret data related to this study, various
% data collection instruments tools are used by research. To that end, one these
% instrument was put into use namely \say{the questionnaire}.


% Two type of questions are used in this study, factual and attitudinal questions.
% In the first part of the survey, factual questions will be used to identify
% some demographic characteristics of the respondents such as their gender, age, academic
% year, department. While attitudinal questions will explore students' attitudes towards
% the use of AI-driven tools in higher education, specifically in the humanities department.


% The questionnaire design predominantly adopts a closed-ended format with predetermined
% response options for participants. However, it also includes four open-ended questions
% to encourage detailed responses and qualitative insights from the participants. These
% open-ended questions focus on soliciting examples of opportunities for integrating AI-driven
% tools in higher education, suggestions for enhancing the integration of AI in higher education,
% sharing experiences of using AI-driven tools, and providing comments or feedback regarding AI-driven initiatives.

\section{Data collection procedure}
Under the mentorship and guidance of my supervisor, Professor Loutfi Ayoub,
a specialized questionnaire was meticulously crafted to engage students within
the Humanities discipline. This survey, aimed specifically at English majors
across various semesters, was disseminated through an array
of social media platforms, with a notable emphasis on WhatsApp.
\section{Data analysis}

As previously mentioned, we gathered data for our recent study through an online questionnaire
sent to the target group. To ensure we had access to a variety of features, we utilized
\say{Google Forms} as our data collection tool. With its ability to accommodate multiple-choice,
checklists, rating scales, and short answer text questions, Google Forms proved to be a
reliable web-based application for our needs.
% The data of the recent study was gathered from the questionnaire that was sent to
% the target group via online platforms as mentioned earlier. Google form was used to collect
% our data since it offers various features to benefit from. In Google form, several types of
% questions can be used such as multiple-choice, checklists, rating scales, and short answers
% text. This web-based application “Google Form” was enough to rely on for data collection.
\section{Conclusion}
This chapter was made to clarify the data collection procedures. A similar effort has been made to
describe data analysis. Therefore, This chapter was designed to lay the groundwork for the upcoming
chapter, tackling the data more in-depth and detailed understanding of the topic.
% This chapter was made to clarifie the data collection procedures. A similar effort has been made
% to describe data anaylsis. Therefore, This chapter was designed to lay the groundwork for the
% upcoming chapter which will tackls the data more in-depth and detailed understand of the topic.
