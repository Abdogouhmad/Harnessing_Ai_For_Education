\chapter{METHODOLOGY}
\section{Introduction}
The present chapter aims to illustrate the research design employed to
investigate the integration of Artificial Intelligence in enhancing
learning experiences in the Humanities. It details the research design,
participants, instruments, data collection, and analysis procedures.
The chosen methodology will be justified for its relevance according
to research questions and hypotheses posted in Chapter 1, section \ref{sec:research-questions-and-hypotheses},
aiming to provide insight into the impact of AI-driven tools in higher education.
\section{Objectives of the study}
This study’s main objective is to examine the effectiveness,
usefulness, and benefits of using AI-driven tools as learning
tools to enhance students’ academic performance. The purpose is to
discover students’ perceptions and the practical ways they use AI
tools to enhance their academic performance. Accordingly,
the following research questions and hypotheses have been
formed based on the objectives.
\section{Research Questions and hypotheses}
\subsection{Research Questions}
The study aims to investigate hressing AI within higher education espacially humanities. Hence,
the following question are formulated:
\begin{itemize}
	\item What are the attitudes of students regarding their academic engagement when utilizing AI-driven tools in educational settings?
	\item What are the attitudes of students toward their academic achievement and performance while employing AI-driven tools?
	\item What are the challenges and opportunities associated
	      with using AI in higher education in Morocco,
	      specifically in the humanities?
\end{itemize}

\subsection{Hypothese}
In terms of achieving the current purpose, the follwing hypotheses have been formulated:
\begin{itemize}
	\item AI-driven tools are singificantly fostering engagment in academic setting.
	\item AI-driven tools are significantly improving academic performance.
	\item There are challenges and opportunities are associated with using AI in higher education
	      in Morocco, specifically in the humanities
\end{itemize}

\section{Research design}
This research employs a mixed-methods approach to explore the
efficacy of AI-driven tools in enhancing learning experiences
in the Humanities. Surveys will be the primary data collection
method to gather insights on effectively integrating AI-driven
tools into the educational process. Furthermore, the surveys will
gather students’ views on their academic performance using these
tools. By combining quantitative survey data with qualitative
student perspectives, this study seeks to provide a comprehensive
analysis of the impact of AI technology on learning outcomes and
student engagement in the Humanities department.
\section{Participants}
The current study investigates the attitudes of one major group,
English university students from different academic years: first year,
second year, and third year. The survey was shared on WhatsApp.
The Participants were asked relevant questions that contributed
to the study. They comprise 60 students from the English department
(44 are female, 16 are male). The age, gender, and previous experience
with AI distribution of this paper’s participants are shown in the table below.

\begin{table}[H]
	\captionof{table}{Participants by Age and Gender}
	\begin{tabular}{l|ccc|ccc|}
		\cline{2-7}
		\multirow{2}{*}{}                & \multicolumn{3}{c|}{Gender}  & \multicolumn{3}{c|}{Age}                                                                                                                                            \\ \cline{2-7}
		                                 & \multicolumn{1}{c|}{female}  & \multicolumn{1}{c|}{male}    & \multicolumn{1}{c|}{prefer not to say} & \multicolumn{1}{c|}{18-25} & \multicolumn{1}{c|}{26-35} & \multicolumn{1}{c|}{36 and above} \\ \hline
		\multicolumn{1}{|l|}{Frequency}  & \multicolumn{1}{c|}{44}      & \multicolumn{1}{c|}{16}      & \multicolumn{1}{c|}{NULL}              & \multicolumn{1}{c|}{39}    & \multicolumn{1}{c|}{12}    & \multicolumn{1}{c|}{9}            \\ \hline
		\multicolumn{1}{|l|}{percentage} & \multicolumn{1}{c|}{73.3 \%} & \multicolumn{1}{c|}{26.7 \%} & \multicolumn{1}{c|}{NULL}              & \multicolumn{1}{c|}{65 \%} & \multicolumn{1}{c|}{20 \%} & \multicolumn{1}{c|}{15 \%}        \\ \hline
	\end{tabular}
\end{table}
% another table is here
\begin{table}[H]
	\begin{tabular}{l|ccc|cc|}
		\cline{2-6}
		\multirow{2}{*}{}                & \multicolumn{3}{c|}{Academic year} & \multicolumn{2}{c|}{The usage of AI for academic purposes}                                                                                             \\ \cline{2-6}
		                                 & \multicolumn{1}{c|}{First Year}    & \multicolumn{1}{c|}{Second Year}                           & \multicolumn{1}{c|}{Third Year} & \multicolumn{1}{c|}{Yes}   & \multicolumn{1}{c|}{No}    \\ \hline
		\multicolumn{1}{|l|}{Frequency}  & \multicolumn{1}{c|}{5}             & \multicolumn{1}{c|}{18}                                    & \multicolumn{1}{c|}{37}         & \multicolumn{1}{c|}{51}    & \multicolumn{1}{c|}{9}     \\ \hline
		\multicolumn{1}{|l|}{percentage} & \multicolumn{1}{c|}{8.3 \%}        & \multicolumn{1}{c|}{30 \%}                                 & \multicolumn{1}{c|}{61.7 \%}    & \multicolumn{1}{c|}{85 \%} & \multicolumn{1}{c|}{15 \%} \\ \hline
	\end{tabular}
	\captionof{table}{Participants by Academic Year and usage of AI for academic purposes}
\end{table}

\begin{Center}
	\begin{figure}
		\centering
		\begin{tikzpicture}
			\pie [rotate = 180, explode = 0.1, text=legend, color = {pink!50, blue!40}]
			{
				73.3/ Females,
				26.7/ males
			}
		\end{tikzpicture}
		\captionof{figure}{Distribution by Gender}
	\end{figure}

	\begin{figure}
		\centering
		\begin{tikzpicture}
			\pie [rotate = 180, explode = 0.1, text=legend, color = {red!50, yellow!40, green!40}]
			{
				65/ 18-25,
				20/ 26-35,
				15/ 36 and above
			}
		\end{tikzpicture}
		\captionof{figure}{Distribution by Age}
	\end{figure}
\end{Center}
% 2nd figires
\begin{Center}
	\begin{figure}
		\centering
		\begin{tikzpicture}
			\pie [rotate = 180, explode = 0.1, text=legend, color = {red!50, magenta!40, gray!40}]
			{
				8.3/ First Year,
				30/ Second Year,
				61.7/ Third Year
			}
		\end{tikzpicture}
		\captionof{figure}{Distribution by Academic year}
	\end{figure}

	\begin{figure}
		\centering
		\begin{tikzpicture}
			\pie [rotate = 180, explode = 0.1, text=legend, color = {cyan!50, purple!40}]
			{
				85/ Yes,
				15/ No
			}
		\end{tikzpicture}
		\captionof{figure}{Distribution by the usage of AI for academic purposes}
	\end{figure}
\end{Center}
\section{Instrument}
\justifying

Various data collection instruments and tools are used by research to collect,
measure, and interpret data related to this study. To that end, one of these
instruments was used, namely an online survey created using
\say{Google Forms}. The survey comprises two types of questions: factual and attitudinal. The factual questions aim to
identify specific demographic characteristics of the participants, such as
their gender, age, academic year, and department. The attitudinal questions
are intended to investigate the students’ attitudes toward utilizing AI-driven
tools in higher education, especially in the humanities department.


The questionnaire design primarily employs a closed-ended format,
which pre-determines response options for the participants. However,
it also incorporates three open-ended questions to elicit detailed
responses and qualitative insights from the participants. These
open-ended questions are designed to encourage the participants
to provide examples of opportunities for integrating AI-driven
tools in higher education, offer suggestions for enhancing the integration
of AI in higher education, share their experiences of using AI-driven tools,
and provide feedback or comments regarding the AI-driven initiatives.
\section{Data collection procedure}
Under the mentorship and guidance of my supervisor, Professor Loutfi Ayoub,
a specialized questionnaire was meticulously crafted to engage students within
the Humanities discipline. This survey, explicitly aimed at
English majors across various semesters at Hassan II University,
was disseminated through various social media platforms,
with a notable emphasis on WhatsApp.
\section{Data analysis}
As mentioned, we gathered data for our recent study through an online
questionnaire sent to the target group. We utilized \say{Google Forms} as
our data collection tool to ensure we had access to various features.
With its ability to accommodate multiple-choice, checklists, rating
scales, and short answer text questions, Google Forms proved to
be a reliable web-based application for our needs.
\section{Conclusion}
This chapter was made to clarify the data collection procedures.
A similar effort has been made to describe data analysis.
Therefore, This chapter was designed to lay the groundwork
for the upcoming chapter, tackling the data with a
more in-depth and detailed understanding of the topic.
