\chapter{METHODOLOGY}
\section{Introduction}
The goal of the present chapter is to illustrate the research design
employeed to to investigate the integration of Artificial Intelligence
in enhancing learning experiences in the Humanities. it details
the research design, participants, instruments, data collection, and
analysis procedures. The chosen methodology will be justified for its
relevance according to research quations adn hypothesis posted in chapter 1 
section: \ref{sec:research-questions-and-hypotheses}, aiming to provide insight
over the impact of AI-driven tools in higher education.
\section{Objectives of the study}
The main of the objective of this study is to examine the effectiveness, usefulness and
benefits of using AI-driven tools as learning tool to enhance the students' academic preformance.
The purpose tends to discover students' presception along with the effective ways they use AI tools
to enhance their academic preformance. Accordingly to the objectives the following research questions
and hypotheses have been formed.
\section{Research Questions and hypotheses}
\subsection{Research Questions}
The study aims to investigate hressing AI within higher education espacially humanities. Hence,
the following question are formulated:
\begin{itemize}
	\item What are the most effective ways to use AI-driven
	      tools for enhancing learning experiences in higher education,
	      especially in the humanities?
	\item What are students’ attitudes toward their academic performance
	      while using AI-driven tools?
	\item What are the challenges and opportunities associated
	      with using AI in higher education in Morocco,
	      specifically in the humanities?
\end{itemize}

\subsection{Hypothese}
In terms of achieving the current purpose, the follwing hypotheses have been formulated:
\begin{itemize}
	\item Students who use AI-driven tools reveal better learning outcomes
	      compared to those who do not in higher education, specifically in the humanities.
	\item AI-driven tools are significantly improving academic
	      performance and engagement in the humanities.
	\item There are challenges and opportunities are associated with using AI in higher
	      education in Morocco, specifically in the humanities.
\end{itemize}

\section{Research design}
In this study, a mixed-methods approach is adopted to investigate the effective utilization 
of AI-driven tools in enhancing learning experiences within the Humanities. Surveys will
be employed as a primary data collection method to gather insights on the ways AI-driven 
tools can be effectively integrated into the educational process. Additionally, the surveys 
will capture students' perceptions of their academic performance when utilizing these tools. 
By combining quantitative data from surveys with qualitative information from student perspectives, 
this study aims to provide a comprehensive analysis of the impact of AI technology on learning outcomes 
and student engagement in the humanities department.
\section{Participants}
still Thinking...
