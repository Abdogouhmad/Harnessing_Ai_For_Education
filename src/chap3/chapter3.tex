\chapter{METHODOLOGY}
\section{Introduction}
The goal of the present chapter is to illustrate the research design
employeed to to investigate the integration of Artificial Intelligence
in enhancing learning experiences in the Humanities. it details
the research design, participants, instruments, data collection, and
analysis procedures. The chosen methodology will be justified for its
relevance according to research quations adn hypothesis posted in chapter 1
section: \ref{sec:research-questions-and-hypotheses}, aiming to provide insight
over the impact of AI-driven tools in higher education.
\section{Objectives of the study}
The main of the objective of this study is to examine the effectiveness, usefulness and
benefits of using AI-driven tools as learning tool to enhance the students' academic preformance.
The purpose tends to discover students' presception along with the effective ways they use AI tools
to enhance their academic preformance. Accordingly to the objectives the following research questions
and hypotheses have been formed.
\section{Research Questions and hypotheses}
\subsection{Research Questions}
The study aims to investigate hressing AI within higher education espacially humanities. Hence,
the following question are formulated:
\begin{itemize}
	\item What are the most effective ways to use AI-driven
	      tools for enhancing learning experiences in higher education,
	      especially in the humanities?
	\item What are students’ attitudes toward their academic performance
	      while using AI-driven tools?
	\item What are the challenges and opportunities associated
	      with using AI in higher education in Morocco,
	      specifically in the humanities?
\end{itemize}

\subsection{Hypothese}
In terms of achieving the current purpose, the follwing hypotheses have been formulated:
\begin{itemize}
	\item Students who use AI-driven tools reveal better learning outcomes
	      compared to those who do not in higher education, specifically in the humanities.
	\item AI-driven tools are significantly improving academic
	      performance and engagement in the humanities.
	\item There are challenges and opportunities are associated with using AI in higher
	      education in Morocco, specifically in the humanities.
\end{itemize}

\section{Research design}
In this study, a mixed-methods approach is adopted to investigate the effective utilization
of AI-driven tools in enhancing learning experiences within the Humanities. Surveys will
be employed as a primary data collection method to gather insights on the ways AI-driven
tools can be effectively integrated into the educational process. Additionally, the surveys
will capture students' perceptions of their academic performance when utilizing these tools.
By combining quantitative data from surveys with qualitative information from student perspectives,
this study aims to provide a comprehensive analysis of the impact of AI technology on learning outcomes
and student engagement in the humanities department.

\section{Participants}
This is the participants we have got from the form

\begin{table}[H]
	\captionof{table}{Participants}
	\begin{tabular}{|l|ccc|ccccc|cc|}
		\hline
		\multirow{3}{*}{} & \multicolumn{3}{c|}{Gender} & \multicolumn{5}{c|}{Age}    & \multicolumn{2}{c|}{Academic Year}                                                                                                                                                                                                                        \\ \cline{2-11}
		                  & \multicolumn{1}{c|}{male}   & \multicolumn{1}{c|}{female} & prefer not to say                  & \multicolumn{1}{c|}{Under 18} & \multicolumn{1}{c|}{18-25} & \multicolumn{1}{c|}{26-35} & \multicolumn{1}{c|}{36-45} & 46-above                 & \multicolumn{1}{c|}{First Year} & \multicolumn{1}{c|}{Second Year} \\ \hline
		Frequancy         & \multicolumn{1}{c|}{}       & \multicolumn{1}{c|}{}       &                                    & \multicolumn{1}{c|}{}         & \multicolumn{1}{c|}{}      & \multicolumn{1}{c|}{}      & \multicolumn{1}{c|}{}      & \multicolumn{1}{c|}{12}  & \multicolumn{1}{c|}{13}                                            \\ \hline
		percentage        & \multicolumn{1}{c|}{}       & \multicolumn{1}{c|}{}       &                                    & \multicolumn{1}{c|}{}         & \multicolumn{1}{c|}{}      & \multicolumn{1}{c|}{}      & \multicolumn{1}{c|}{}      & \multicolumn{1}{c|}{5\%} & \multicolumn{1}{c|}{10\%}                                          \\ \hline
	\end{tabular}
\end{table}



\section{Instrument}
\justifying
To collect, measure, and interpret data related to this study, various
data collection instruments tools are used by research. To that end, one these
instrument was put into use namely \say{the questionnaire}.



Two type of questions are used in this study, factual and attitudinal questions.
In the first part of the survey, factual questions will be used to identify
some demographic characteristics of the respondents such as their gender, age, academic
year, department. While attitudinal questions will explore students' attitudes towards
the use of AI-driven tools in higher education, specifically in the humanities department.



The questionnaire design predominantly adopts a closed-ended format with predetermined
response options for participants. However, it also includes four open-ended questions
to encourage detailed responses and qualitative insights from the participants. These
open-ended questions focus on soliciting examples of opportunities for integrating AI-driven
tools in higher education, suggestions for enhancing the integration of AI in higher education,
sharing experiences of using AI-driven tools, and providing comments or feedback regarding AI-driven initiatives.
\section{Data collection procedure}
With the help and guide of my supervisor professor Loutfi Ayoub, the questionnaire was created to target Humanities students.
the survey was sent to many humanities student department with different semsters through social media platform such as WhatsApp.
\section{Data analysis}
The data of the recent study was gathered from the questionnaire that was sent to
the target group via online platforms as mentioned earlier. Google form was used to collect
our data since it offers various features to benefit from. In Google form, several types of
questions can be used such as multiple-choice, checklists, rating scales, and short answers
text. This web-based application “Google Form” was enough to rely on for data collection.
