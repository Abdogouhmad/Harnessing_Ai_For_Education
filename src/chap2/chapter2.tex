\chapter{LITERATURE REVIEW}\label{ch:literature-review}
\section{Introduction}\label{sec:introduction}
\justifying
Before discussing the study of ``Harnessing Artificial Intelligence for Enhanced
Learning Experiences in the Humanities'', it is imperative to first include a review of the most
important observations and viewpoints on the topic.
This chapter provides an overview of
the implementation or integration of AI in high education, focusing on practical ways of using AI-driven tools
to enhance academic performance and productivity.
Additionally,
this chapter addresses the challenges and opportunities associated with the use of AI in higher education
institutions in Morocco and abroad.
Finally, the chapter concludes with users' perceptions, which are consolidated with
statistics and studies conducted by researchers.
The aim is to clarify
what has been uncovered about this topic through the vivid opinions of users.

\section{Defining Key Concepts}\label{sec:defining-key-concepts}
\begin{itemize}
	\item \textbf{Artificial Intelligence (AI)}\label{AI} refers to the ability of a computer system to perform human
	tasks that can be accomplished by human Intelligence\citep{sadiku_ai_2021}.
%	2nd diff
	\item \textbf{AI-driven tools in education} encompass the application of AI tools like \say{ChatGPT} to assist 
	students, educators and administration in an education process.
	These AI-driven tools are used for planning and reactive execution of educational phases, such as
	student admission, lesson planning, knowledge delivery and performance evaluation\citep{mallik_proactive_2023}.
	Additionally, it serves as an extension of human intelligence, enabling
	increased productivity in the educational sphere by performing tasks
	such as problem-solving, learning, and decision-making\citep{cheng_widespread_2023}.
%	3rd diff
	\item \textbf{Learning experience in higher education}  refers to designing and implementing educational activities to create 
	positive and foster engaging student learning experiences \citep{kang_supporting_2023}.
	it involves comprehending and assessing the students’ educational experience, including 
	their satisfaction, self-efficacy, engagement, and self-regulated learning experience\citep{lyz_students_2022}.
	The focus is on improving the quality of education by enhancing students’ academic success, 
	readiness for self-education and self-development, and subject well-being \citep{iordache-platis_building_2018}.
% 4rd diff
	\item \textbf{Intelligent Tutoring Systems}
\end{itemize}
\section{The use of AI in Higher Education }\label{use-ai}
\justifying
Artificial intelligence has been increasingly integrated into various aspacts of higher
education, transforming traditional eduction \citep{wang_exploring_2023}. This section explores
some ways that AI can be used to enhance learning experiences and increas the academic studens' performance
by focusing on  personalized learning, intelligent tutoring, and administrative tasks automation.

\subsection{Personalized Learning}
Personalized learning wih AI in higher education has the potential to enhance the academic peroformance and engagment
by providing tailored learning experiences for students. Using AI technologies in education helps identifying patterns in student performance and preferences 
through algorithms and data analysis. This enables personalized content and activity recommendations, enhancing the students' 
learning experience, motivation, and engagement \citep{guerrero-quinonez_artificial_2023}. Also, it provides a tailored resources based on spacific 
needs and learning styles. In addition, It can track and analysize progress in real-time, indetifying the gap areas that needs more support and adjusting the learining
materials accordingly \citep{l_d_of_cs_akshara_first_grade_college_2023}.

\subsubsection{Intelligent Tutoring Systems (ITS) as module of Perosnalized learning}
Intelligent Tutoring Systems (ITS) offers a promising approach to enhance online learning
with the help of AI, providing personalized support, instant feedback, 
and continuous monitoring for more effective and autonomous learning. it uses AI algorithms anaylze students' data,
enabling personalized experience. These systems are adaptative to the indiviual needs of students,
offering relevent content and personalized feedback \citep{l_d_of_cs_akshara_first_grade_college_2023}. these
systems improves the adaptiveness and laverage personalized learning by considering the individual needs of 
each students \citep{bradac_design_2022}.
\subsection{ChatBots ``ChatGPT'' as a module}
hello
