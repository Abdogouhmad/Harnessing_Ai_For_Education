\chapter{LITERATURE REVIEW}\label{ch:literature-review}
\section{Introduction}\label{sec:introduction}
\justifying
Before discussing the study of ``Harnessing Artificial Intelligence for Enhanced
Learning Experiences in the Humanities'', it is imperative to first include a review of the most
important observations and viewpoints on the topic.
This chapter provides an overview of
the implementation or integration of AI in high education, focusing on practical ways of using AI-driven tools
to enhance academic performance and productivity.
Additionally,
this chapter addresses the challenges and opportunities associated with the use of AI in higher education
institutions in Morocco and abroad.
Finally, the chapter concludes with users' perceptions, which are consolidated with
statistics and studies conducted by researchers.
The aim is to clarify
what has been uncovered about this topic through the vivid opinions of users.

\section{Defining Key Concepts}\label{sec:defining-key-concepts}
\begin{itemize}
	\item \textbf{Artificial Intelligence (AI)}\label{AI} refers to the ability of a computer system to perform human
	tasks that can be accomplished by human Intelligence\citep{sadiku_ai_2021}.
%	2nd diff
	\item \textbf{AI-driven tools in education} encompass the application of AI tools like \say{ChatGPT} to assist 
	students, educators and administration in an education process.
	These AI-driven tools are used for planning and reactive execution of educational phases, such as
	student admission, lesson planning, knowledge delivery and performance evaluation\citep{mallik_proactive_2023}.
	Additionally, it serves as an extension of human intelligence, enabling
	increased productivity in the educational sphere by performing tasks
	such as problem-solving, learning, and decision-making\citep{cheng_widespread_2023}.
%	3rd diff
	\item \textbf{Learning experience in higher education}  refers to designing and implementing educational activities to create 
	positive and foster engaging student learning experiences \citep{kang_supporting_2023}.
	it involves comprehending and assessing the students’ educational experience, including 
	their satisfaction, self-efficacy, engagement, and self-regulated learning experience\citep{lyz_students_2022}.
	The focus is on improving the quality of education by enhancing students’ academic success, 
	readiness for self-education and self-development, and subject well-being \citep{iordache-platis_building_2018}.
% 4rd diff
	\item \textbf{Intelligent Tutoring Systems}
	hello
	\item \textbf{ ChatBots }\label{chatbot} are computer programs that replicate human conversation with a conclusion. 
	While not all chatbots possess AI capabilities, modern chatbots are progressively 
	integrating AI techniques to analyze human input\citep{IBM_withnodate}.
	This enables the digitization of human interaction through written 
	or vocal means, giving the impression of ongoing communication with another individual \citep{oracle_what_nodate}.
	\item \textbf{Education Data Mining (EDM)}  is a technique used to measure students' academic achievement, 
	assess the learning process, evaluate the overall quality of education, and improve the results in higher education.
	It involves processing and analyzing large amounts of data to extract useful 
	information for decision-making and policy-making in the field of education \citep{arifin_using_2022}.
\end{itemize}
\section{The use of AI in Higher Education }\label{use-ai}
\justifying
Artificial intelligence has been increasingly integrated into various aspacts of higher
education, transforming traditional eduction \citep{wang_exploring_2023}. This section explores
some ways that AI can be used to enhance learning experiences and increas the academic studens' performance
by focusing on  personalized learning, intelligent tutoring, and administrative tasks automation.

\subsection{Personalized Learning}
The use of AI technologies in higher education for personalized learning has been shown to enhance academic 
performance and engagement by providing tailored learning experiences for students. Through algorithms and data analysis, 
AI can identify patterns in student performance and preferences, enabling personalized content and activity recommendations. 
This, in turn, improves the student’s learning experience, motivation, and engagement. Additionally, AI can provide tailored 
resources based on specific needs and learning styles and track real-time progress, 
identifying areas requiring more support and adjusting the learning materials accordingly 
\citep{guerrero-quinonez_artificial_2023} and \citep{l_d_of_cs_akshara_first_grade_college_2023}.


% Personalized learning wih AI in higher education has the potential to enhance the academic peroformance and engagment
% by providing tailored learning experiences for students. Using AI technologies in education helps identifying patterns in student performance and preferences 
% through algorithms and data analysis. This enables personalized content and activity recommendations, enhancing the students' 
% learning experience, motivation, and engagement \citep{guerrero-quinonez_artificial_2023}. Also, it provides a tailored resources based on spacific 
% needs and learning styles. In addition, It can track and analysize progress in real-time, indetifying the gap areas that needs more support and adjusting the learining
% materials accordingly \citep{l_d_of_cs_akshara_first_grade_college_2023}.

\subsubsection{Intelligent Tutoring Systems (ITS) as module of Perosnalized learning}

Intelligent Tutoring Systems (ITS) offer a promising approach to enhance online learning with the help of AI. 
ITS provides personalized support, instant feedback, and continuous monitoring for more effective and autonomous learning. 
It uses AI algorithms to analyze students' data, enabling personalized experiences. These systems adapt to each student's needs, 
offering relevant content and personalized feedback. According to \citep{l_d_of_cs_akshara_first_grade_college_2023}, these systems improve adaptiveness and leverage 
personalized learning by considering the individual needs of each student. \citep{bradac_design_2022} also support this approach 
of leveraging personalized learning to enhance students' learning experience.


% Intelligent Tutoring Systems (ITS) offers a promising approach to enhance online learning
% with the help of AI, providing personalized support, instant feedback, 
% and continuous monitoring for more effective and autonomous learning. it uses AI algorithms anaylze students' data,
% enabling personalized experience. These systems are adaptative to the indiviual needs of students,
% offering relevent content and personalized feedback \citep{l_d_of_cs_akshara_first_grade_college_2023}. these
% systems improves the adaptiveness and laverage personalized learning by considering the individual needs of 
% each students \citep{bradac_design_2022}.
\subsection{ChatBots ``ChatGPT'' as a module}

In the field of higher education, ChatGPT \footnote{Chatgpt is not only the Ai-driven tools that are in market yet some 
of them are based on their API
} has proven its value as a sophisticated AI-driven tool powered by OpenAI. 
Based on the students' learning history, it offers personalized recommendations. In addition, ChatGPT can provide 
accurate answers to questions with minimal input. This assists students to improve their study skills and time 
management and increases motivation and engagement with learning by providing access to a wide range of resources. It can
be used to assess students' writing \citep{mohammed_exploring_2023}. 
ChatGPT can effectively shape teaching by allowing educators to make the right decisions, 
offering personalized assitance and support outside of regular class hours. It also prompts engagement and active 
learning through interactive and dynamic learning experiences. It can facilitate discussion, stimulate critical thinking, 
and provide immediate feedback, contributing to a better learning experience \citep{schonberger_chatgpt_2023}. 
ChatGPT has the potential to enhance learning and teaching processes, providing benefits in class 
preparation, exam preparation, and as a personal tutor \citep{domenech_chatgpt_2023}. 

% old
% ChatGPT is a sophisticated AI-powered chatbot \footnote{ChatBots are computer programs that imitate human conversation. 
% They may not all have AI, but modern ones are using AI techniques to understand input. 
% ChatBots allow digital interaction to mimic real communication with humans.} powerd by \href{https://openai.com/}{Openai},
% has emerged as a valuable tool in the higher education landscape. It offers personalized 
% recommendation based on the students' learining history. In addition, ChatGPT can provide
% accurate answer to question with minimal input assisting students to improve their study skills
% and time managment and increase motivation and engagement with learning by providing access to a
% wide range of resources \citep{mohammed_exploring_2023}. ChatGPT can effectively shape
% teaching allowing educators to make the right decisions, providing personalized assitance and 
% support outside of regular class hours. it also prompt engagement and active learning
% roviding interactive and dynamic learning experiences. It can facilitate discussions, 
% prompt critical thinking, and offer immediate feedback, enhancing the overall learning process \citep{schonberger_chatgpt_2023}.
% ChatGPT has the potential to enhance learning and teaching processes, 
% providing benefits in class preparation, exam preparation, and as a personal tutor \citep{domenech_chatgpt_2023}.


\subsection{AI and Administrative Efficiency: Streamlining Operations}
AI can be effectively be used to streamline the administrative tasks and automating them, allowing educators
to focus more on important activities such as curriculum design \citep{drach_use_2023}. It can streamline enrollment
and improve retention, providing institutional possibilities for better 
resource management and successful online training processes \citep{lukianets_promises_2023}. Additionally, AI enables data analysis and 
pedagogical reporting, facilitating evidance-based decision-making \citep{guerrero-quinonez_artificial_2023}.

\subsubsection{Educational Data Mining (EDM) as a module}
EDM is a technique used to extract the knowledge from data and improve academic performance, learning quality, and decision-making. 
It involves the use of statistical analysis, machine learning, and data mining to analyze academic, socioeconomic, 
and learning analytics data \citep{hooda_integrating_2022, arifin_using_2022}. EDM has shown promising results in predicting students' 
performance, with techniques like J48 \footnote{J48 is a decision tree algorithm that is commonly used in 
educational data mining (EDM) for classification tasks.} and K-means \footnote{ K-means is a clustering algorithm that is often used in educational data mining 
(EDM) for grouping similar data points together.  
It is an unsupervised learning algorithm that aims to partition the data into K clusters, where K is a predefined number. 
The algorithm iteratively assigns data points to the nearest cluster centroid and updates the centroids until convergence. 
K-means is widely used in EDM for analyzing student behaviors and identifying patterns in educational data.} being effective \citep{prince_sattam_bin_abdulaziz_university_state_2016}. 
The use of Data Mining methods in education, specifically EDM, has the potential to enhance the overall efficiency and success of educational institutions.
