\chapter{LITERATURE REVIEW}\label{ch:literature-review}
\section{Introduction}\label{sec:introduction}
\justifying
Before discussing the study of ``Harnessing Artificial Intelligence for Enhanced
Learning Experiences in the Humanities'', it is imperative to first include a review of the most
important observations and viewpoints on the topic.
This chapter provides an overview of
the implementation or integration of AI in high education, focusing on practical ways of using AI-driven tools
to enhance academic performance and productivity.
Additionally,
this chapter addresses the challenges and opportunities associated with the use of AI in higher education
institutions in Morocco and abroad.
Finally, the chapter concludes with users' perceptions, which are consolidated with
statistics and studies conducted by researchers.
The aim is to clarify
what has been uncovered about this topic through the vivid opinions of users.

\section{Defining Key Concepts}\label{sec:defining-key-concepts}
\begin{itemize}
	\item \textbf{Artificial Intelligence (AI)}\label{AI} refers to the ability of a computer system to perform human
	tasks that can be accomplished by human Intelligence\citep{sadiku_ai_2021}.
%	2nd diff
	\item \textbf{AI-driven tools in education} encompass the application of AI tools like \say{ChatGPT} to assist 
	students, educators and administration in an education process.
	These AI-driven tools are used for planning and reactive execution of educational phases, such as
	student admission, lesson planning, knowledge delivery and performance evaluation\citep{mallik_proactive_2023}.
	In addition, it serves as an extension of human intelligence, which
	increases productivity in the educational sphere by performing tasks
	such as problem-solving, learning, and decision-making\citep{cheng_widespread_2023}.
%	3rd diff
	\item \textbf{Learning experience in higher education} refers to the design
	and implementation of educational activities that aims to create positive and foster engaging learning experience for students\citep{kang_supporting_2023}.
	it involves comprehending and assessing the students' educational experience, including
	their satisfaction, self-efficacy, engagement, and self-regulated learning experience\citep{lyz_students_2022}.
	The focus is on improving the quality of education by enhancing students' academic success, readiness for self-education and self-development
	and subject-well-being\citep{iordache-platis_building_2018}.
\end{itemize}
\section{The Global Landscape: Surveying International Research on AI in Higher Education}\label{ai-in-world}
hello
