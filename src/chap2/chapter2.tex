\chapter{LITERATURE REVIEW}
\section{Introduction}
\justifying
it is imperative first to appraise the most significant
insights and perspectives. This chapter presents a comprehensive survey of how AI has been
incorporated into higher education, emphasizing practical methods for utilizing AI-based
resources to improve academic achievement and efficiency.
The chapter concludes by addressing the obstacles that arise when implementing AI in educational
institutions. The objective is to elucidate the findings surrounding this subject matter, gleaned from other
researchers' diverse viewpoints.

\section{Defining Key Concepts}
\subsection{Artificial Intelligence (AI)}
Artificial Intelligence is the intelligence demonstrated by machines,
involving tasks such as speech recognition, computer vision,
and language translation.  Machines can perceive, synthesize, and infer information,
distinguishing it from the intelligence displayed by non-human animals and humans \citep{ola_artificial_2023}.
In addition, AI has the ability of  a computer system
to perform human tasks that
human intelligence can accomplish \citep{sadiku_ai_2021}.
It has the potential to optimize and improve various ascpacts
of human life, including education, health, business,
and technology \citep{cheng_widespread_2023}.
%	2nd diff
\subsection{AI-driven tools in education}
AI-driven tools encompass the application of AI tools such as \say{ChatGPT} to assist students,
educators and administration in an education process. These AI-driven tools are used for
planning and reactive execution of educational phases, such as student admission, lesson planning,
knowledge delivery and performance evaluation \citep{mallik_proactive_2023}.
Additionally, it serves as an extension of human intelligence, enabling increased productivity in
the educational sphere by performing tasks such as problem-solving, learning, and decision-making\citep{cheng_widespread_2023}.
%	3rd diff
\subsection{Learning experience in higher education}
It refers to designing and implementing educational activities to create positive
and foster engaging student learning experiences\citep{kang_supporting_2023}.
It involves comprehending and assessing the students’ educational experience, including their satisfaction,
self-efficacy, engagement, and self-regulated learning experience\citep{lyz_students_2022}.
The focus is on improving the quality of education by enhancing students’ academic success,
readiness for self-education and self-development, and subject well-being\citep{iordache-platis_building_2018}.
% 4rd diff
\subsection{Intelligent Tutoring Systems}
ITS is educational software that incorporates AI. The software monitors students’ progress,
adjusts feedback, and provides hints to offer personalized guidance\citep{shute_intelligent_2010}. It aims to provide individualized, sophisticated instructional advice
\citep{sedlmeier_intelligent_2001}.
\subsection{ ChatBots }
Chatbots are computer programs that replicate human conversation with a conclusion.
While not all chatbots possess AI capabilities, modern chatbots are progressively
integrating AI techniques to analyze human input\citep{IBM_withnodate}.
It enables the digitization of human interaction through written or vocal means,
giving the impression of ongoing communication with another individual \citep{oracle_what_nodate}.
\subsection{Education Data Mining (EDM)}
EDM is a technique that is used to evaluate students' academic performance,
assess the learning process, determine the overall quality of education, and enhance outcomes in higher education.
It entails processing and analyzing large amounts of data to extract relevant information that can be used for
decision-making and policy-making in the education sector \citep{arifin_using_2022}.

% the section
\section{The use of AI in Higher Education for enhancing academic performance and engagement }
\justifying
Artificial intelligence has been increasingly integrated into various aspacts of higher
education, transforming traditional education \citep{wang_exploring_2023}.  This section explores ways
AI can enhance learning experiences and increase academic students' performance by focusing on AI-Assessment,
personalized learning, intelligent tutoring, and administrative task automation.
\subsection{AI-Assessment}
\citep{crompton_artificial_2023} conducted a systematic review of 138 articles from 2016-2022. 
The study provides unique findings of an examination of who the AIEd 
was intended for and majority were students with 72\% 
answering the overarching question of how AIEd was used in higher education. 
Assessment and evaluation was one of five uses that students use. 26 studies 
clarify that automated assessment is the most commonly used for academic achievement. 
\citep{zhang_student_2022} used automation assessment to improve the 
academic writing skills of Uyghur ethnic minority students living in China. 
The study found that due to the diverse cultural aspects of writing, 
students interacted with the automated assessment system in terms of 
behavior, cognition, and emotions. This interaction facilitated their 
self-directed learning process and contributed to their improvement of their 
academic performance, namely writing skill.

\subsection{Personalized Learning}
The application of AI technology in higher education has been found to enhance academic performance
and engagement by providing personalized learning experiences for students. By utilizing algorithms
and data analysis, AI can recognize patterns in student performance and preferences, leading to personalized
content and activity suggestions. This, in turn, enhances the student’s learning experience, motivation, and
engagement. Furthermore, AI can offer customized resources based on individual needs and learning styles while
monitoring real-time progress to identify areas that require additional support and adjusting learning materials
accordingly \citep{guerrero-quinonez_artificial_2023} and \citep{l_d_of_cs_akshara_first_grade_college_2023}.

\subsection{Intelligent Tutoring Systems (ITS)}

Intelligent Tutoring Systems (ITS) have shown great promise in enhancing online learning through  AI.
They provide personalized support, immediate feedback, and ongoing monitoring for more effective and independent learning.
By analyzing student data with AI algorithms, these systems deliver tailored experiences that adapt to each student's needs,
offering relevant content and personalized feedback. According to a recent study by \citep{l_d_of_cs_akshara_first_grade_college_2023},
it improves adaptiveness and leverages personalized learning by considering each student's individual needs.
This approach to personalized learning is also supported by \citep{bradac_design_2022} who believe it can greatly
improve students' learning experiences.

\subsection{ChatBots ``ChatGPT'' as a module}

ChatGPT has become a valuable tool in higher education, providing students personalized recommendations
based on their learning history. With minimal input, it can accurately answer questions and assist students
in improving their study skills and time management. Additionally, it motivates and engages students by
offering access to many resources. ChatGPT can assess students' writing abilities\citep{mohammed_exploring_2023}.
Moreover, ChatGPT is an effective teaching aid that enables educators to make informed decisions and provides
personalized support outside regular class hours. It also promotes engagement and active learning through interactive
and dynamic experiences, facilitating discussions, stimulating critical thinking, and delivering immediate feedback
to enhance the learning experience\citep{schonberger_chatgpt_2023}.
Ultimately, ChatGPT has the potential to enhance both learning and teaching processes,
serving as an invaluable tool for class preparation, exam preparation, and personalized tutoring\citep{domenech_chatgpt_2023}.



\subsection{AI and Administrative Efficiency: Streamlining Operations}
Artificial intelligence has proven to be a valuable tool in enhancing administrative processes for educators.
By automating tasks, educators can prioritize important activities such as curriculum design \citep{drach_use_2023}.
Furthermore, AI can streamline enrollment and improve retention rates, offering opportunities for resource optimization
and successful online training experiences \citep{lukianets_promises_2023}. In addition, AI's data analysis capabilities
and pedagogical reporting facilitate evidence-based decision-making, empowering educators to make informed choices\citep{guerrero-quinonez_artificial_2023}.

\subsection{Educational Data Mining (EDM)}

Educational Data Mining (EDM) is a powerful tool for extracting knowledge from academic, socioeconomic,
and learning analytics data. EDM can significantly improve academic performance, learning quality,
and decision-making by utilizing statistical analysis, machine learning, and data mining techniques \citep{hooda_integrating_2022, arifin_using_2022}.
Recent studies have highlighted the effectiveness of EDM in predicting students' performance using
practical techniques such as J48 \footnote{J48 is a decision tree algorithm that is commonly used in
	educational data mining (EDM) for classification tasks.} and K-means.\footnote{ K-means is a clustering algorithm that is often used in educational data mining
	(EDM) for grouping similar data points together. It is an unsupervised learning algorithm that aims to partition the data into K clusters, where K is a predefined number.
	The algorithm iteratively assigns data points to the nearest cluster centroid and updates the centroids until convergence.
	K-means is widely used in EDM for analyzing student behaviors and identifying patterns in educational data.}
With its potential to enhance overall efficiency and success\citep{prince_sattam_bin_abdulaziz_university_state_2016},
the use of Data Mining methods, particularly EDM, is becoming increasingly essential for educational institutions.


\section{Challenges of AI in higher education}
The emergence of artificial intelligence (AI) and its growing utilization in educational
contexts have brought numerous challenges accompanying its implementation.  This section explores the obstacles
and issues when integrating AI into education settings, particularly its use for academic purposes.
\subsection{AI bias}
Concerns have been raised regarding the potential negative impact of using AI in admission or
grading processes for students. AI algorithms can produce racially biased output when trained on biased data.
For example, in medical appointment scheduling, certain algorithms predict that black patients are more likely
to miss appointments compared to non-black patients. This perpetuates racial inequalities and creates a lack of
access to healthcare, highlighting the crucial need for accuracy and fairness in AI. The implications of this
extend beyond the medical field and into other domains such as education, judicial systems, and public safety\citep{shanklin_ethical_2022}.
The facial detection algorithms used by the software may be biased against students based on their skin tone or gender.
The study shows that students with darker skin tones and black students are more likely to be marked for review, and women
with darker skin tones are selected for review more often than white men. This highlights the need for caution when using
automated proctoring software, as biased AI algorithms can significantly affect education, social justice, equity, and diversity
\citep{yoder-himes_racial_2022}.

\subsection{Data privacy}
Schools, teachers, and students generate vast amounts of private data in digital formats,
but often have little control over this data, which is instead held by third-party institutions. This
lack of autonomy poses challenges in protecting student privacy. The use of AI systems and big
data analytics in education has significantly enhanced the capacity to identify information about
student users. This includes data traces from online learning activities, such as browsing history,
download history, and location data. The security boundary of personal information privacy is 
becoming increasingly blurred, posing challenges in protecting student data privacy \citep{huang_ethics_2023}. 
It is important to address the issue  of data privacy in artificial intelligence (AI) within higher 
education. AI technology aggregates a vast amount of data from various subfields, 
making it crucial for data processing and consumption to adhere to privacy and security principles.
With the advent of the Internet, the retention period  of information has significantly increased. 
Therefore, data for AI systems must be collected, utilized,  shared, stored, and deleted under information 
security standards. Protecting personal information related to the lifespan of AI technology
should be ensured by legal frameworks and ethical norms\citep{unesco_2022}.

\section{Conclusion}
This literature review comprehensively explored the burgeoning integration of Artificial Intelligence (AI) in higher education.
AI offers a multitude of benefits, including personalized learning pathways through intelligent tutoring systems and chatbots,
streamlined administrative tasks via automation, and data-driven decision-making empowered by educational data mining (EDM).
These advancements hold immense promise for a future characterized by optimized learning experiences and efficient institutional
operations. However, challenges such as algorithmic bias and student data privacy necessitate careful consideration.
To ensure AI's ethical and equitable implementation, future endeavors should focus on mitigating bias within algorithms and developing
robust data security protocols. In essence, AI presents a powerful but nuanced tool for transforming higher education.
By navigating the ethical and practical hurdles and harnessing its full potential, AI can revolutionize learning experiences
and empower educators to cultivate a more effective and equitable educational landscape.

