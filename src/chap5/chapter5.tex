\chapter{GENERAL CONCLUSION}
\section{Introduction}

This chapter offers a general overview of this research paper.
It contains a restatement of the current study’s objectives, a summary
of the methodology and findings obtained from the investigation, and 
it provides limitations and suggestions for further studies.

\section{Summary of the Research Goal}

This paper is an attempt to explore university students’ attitudes towards the 
use of AI as a tool to enhance the learning experiences within the humanities, 
namely improving academic performance and engagement. In addition, exploring the
challenges and opportunities faced by humanities students(the study focuses on English university students).

\section{Summary of the Findings}


The present study is based on quantitative and qualitative data collection. 
It indicates that Moroccan English students at the University of Hassan II and the faculty of Ben Msik have
a positive attitude towards using AI to improve their learning experiences. 
The majority of respondents believe that their academic performance has improved since using AI.
The findings suggest that AI can enhance personalized learning and enable researchers to analyze their data accurately. 
Additionally, AI helps students to remain moderately engaged in the classroom
despite challenges that may hinder their preference for using AI for academic purposes. 
These findings are consistent with previous studies and demonstrate the benefits of using AI in an academic context. 
Furthermore, the responses from participants confirmed the research hypotheses, which are:
\begin{itemize}
	
\item AI-driven tools significantly promote engagement in academic settings.
\item AI-driven tools significantly improve academic performance.
\item There are challenges and opportunities associated with using AI in higher education in Morocco, especially in the humanities.
\end{itemize}
\section{Limitations of the study}
This study investigated the feasibility of using AI-driven tools in the academic context to enhance learning experiences.
It was particularly interesting to see whether AI-driven tools promote engagement and improve academic performance.
The current research provided evidence for the effectiveness of achieving an enhanced learning experience in higher education.
 However, it is imperative to acknowledge certain limitations inherent in this study. 
Foremost, the absence of professors' perspectives constitutes a significant gap, 
given their pivotal role within the educational ecosystem. In addition,
the scope of this study does not encompass a diverse array of university 
students across different disciplines, thereby limiting the generalizability 
of its findings. To gain a more comprehensive understanding, 
it is crucial to gather additional data 
from participants across a wider range of Moroccan universities before making any generalizations.
\section{Implications of the Study}

The findings of the current study and previous research clearly demonstrate that AI can
improve learning experiences in higher education.
This can lead to increased student engagement and improved academic performance.
Given these findings, it is recommended that educators and learners incorporate this valuable
technology into teaching and learning practices. The use of AI technology can benefit humanities
students, particularly those studying English, by enhancing their listening, speaking, reading, 
and writing skills, as well as expanding their vocabulary. Additionally, educators can utilize 
AI to automate routine tasks, allowing them to focus on core objectives and deliver effective lesson plans.

\section{Suggestions for further research}

The use of AI in higher education to enhance learning experiences is a newly emerging area.
Further study will contribute to its survival and development.
Therefore, future research should broaden the scope by collecting more data 
from different universities in various regions of Morocco. 
In addition, it is important to consider the perspectives of the professors to obtain more truthful and accurate findings.

\section{Conclusion}

The 21st century has seen a technological revolution, with the widespread availability
of devices used for communication, entertainment, and education. This research paper
aims to examine the effectiveness of AI-driven tools, such as mobile and web applications,
in improving learning experiences within Moroccan higher education institutions. 
This chapter provides a conclusion to the main points explored in the study, 
including a summary of the research goals, methodology, findings, suggestions 
for further research, as well as the limitations and implications of the study.

