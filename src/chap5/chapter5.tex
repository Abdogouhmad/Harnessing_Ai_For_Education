\chapter{GENERAL CONCLUSION}
\section{Introduction}
This chapter offer a general overiew of this research paper.
It contains a restatement of the current study's objectives, a summary of
the methodology and findings obtained from the investigation, as well as it provides
limitations, and suggestion for further studies.

\section{Summary of the Research Goal}
This paper is an attempt to explore university students’ attitudes towards the use of
AI as tool to enhance the learning experiences within the humanities, namely improving
academic performance and engagments. In addition exploring the challages and oppurtunities
faced by humanities students(the study focuses on English university students).

\section{Summary of the Findings}
The findings of the present study are drawn from statistical measures since a
quantitative data collection has been used. They depict
that Moroccan English student university, namely the university of Hassan II, faculity of Ben
Msik have a positive attitude toward using AI to enhance their learning experiences.
The vast majority of respondents precieve their academic preformace is improved since using
AI. According to the findings, AI can enhance personalized learning and enabling researchers
to analysize their data accurntly. In addition to this, AI helps them to be moderately engaged
in classroom regardless the challenges faced that imped their preferance using AI for academic
purposes. Hence, the finding are further align with those of \citep{mohammed_exploring_2023},
\citep{lyz_students_2022}, \citep{guerrero-quinonez_artificial_2023}, \citep{l_d_of_cs_akshara_first_grade_college_2023},
and many other studies in the field that demonstrate the profitablity of using AI within academic context.
Finally, the responses given by the participants confirmed the research
hypotheses mentioned in the methodology chapter which were as follows:
\begin{itemize}
	\item AI-driven tools are singificantly fostering engagment in academic setting.
	\item AI-driven tools are significantly improving academic performance
	\item There are challenges and opportunities are associated with using AI in higher education␍
	      in Morocco, specifically in the humanities
\end{itemize}

\section{Limitations of the study}
This study looked at the feasibility of using AI-driven tools in academic context as tool
for enhancing the learning experiences. It was particularly interested in whether AI-driven
tools fostering engagment and improving academic performance. The current research presented
evidence for the efficacy to achieve an enhanced learning experience in higher education.
Neverthless, this study has some limitations that must be mentioned. Firstly, paper adopted
an online data collection method, namely “the questionnaire”. Our reliance
on online data collection was because of the quarantine which unfortunately led to the lack
of face-to-face interviews to probe. Yet, interviews often result in more accurate and
reliable data. Secondly, this study lacks the professor’s perspectives for more accurate
findings as they are an integral part of the educational system. Finally, this study does not
include all universities students from diffrenet desciplines. Thus, the findings cannot be generalized. More data
from participants from various Moroccan universities must be collected before any
generalization.

\section{Implications of the Study}
The current study’s findings and former studies vividly show that AI can enhance learning
experiences in Moroccan higher education, namely in the humanities. Making academic students
more engaged and improved their academic preformance. In light
of such findings and the foregoing studies, it is recommended that educators and learners
integrate this valuable technology in teaching and learning since the current
study and former studies showed that this technology could enhance learning experience.
Otherwise, this technology may help humanities students, namely English students to improve their
listening, speaking \footnote{ChatGPT new update enable users to chat using voice chat also Gemini is integrated in place of google voice assistence.}
,reading, and writing skills as much as it can help them increase their
vocabulary repertoire. At the same time, educators may use this technology to automate their minor tasks that can take
a vast amount of time. Making them more focused on the core tasks that can enable them to deliver a good lesson plan
that target their objectives.
\section{Suggestions for further research}
The use of AI in higher education for enhancing learning experineces is a newly emerging area.
Any further study will contribute its survival and development. Therefore, it would be more fitting
for future research to broaden the scope of the study by collecting more data from different
universities in different regions of Morocco. Besides, the professor’s perspectives must be
taken into account for more truthful and accurate findings.
\section{Conclusion}
The 21st century has witnessed a technological revolution, with the widespread 
availability of technological devices that are increasingly used for 
communication, entertainment, and educational purposes. In this regard, this research paper aimed to look at the efficacy of
AI-driven tools, as a mobile and web application, for enhancing learning experiences within
Moroccan higher education institutions. The focus of
this chapter has been to provide a finalizing conclusion to the main points explored in this
study. Accordingly, this chapter focused on a synopsis of the research goals, methodology,
and findings. It also provides suggestions for further research, as well as the limitations and
implications of the study.
