\chapter{INTRODUCTORY CHAPTER}\label{ch:introductory-chapter}
%\addcontentsline{toc}{chapter}{INTRODUCTORY CHAPTER}


\section{Problem statement}\label{sec:problem-statement}
\justifying
Artificial Intelligence is revolutionized, and its potential
to transform higher education is increasingly appreciated.
However, despite the emergency of artificial intelligence and
AI-driven tools like \say{ChatGPT}which has significant capabilities,
there remains a gap in comprehending how effectively Ai can
enhance learning experiences within humanities.
While \say{ChatGPT}
and similar AI-driven tools have gained prominence across various
industries since its release in late November 2022, it has not been
fully utilized in the field of education.
Rather than serving users
to accomplish their tasks for better productivity.
As a result, questions
have been raised about the results in fostering genuine education engagement
and knowledge acquiring among a student with the existence of AI-driven tools.
Therefore, the central problem of this study is How can Ai be harnessed
in education to enhance learning experiences in the humanities.
% 2nd section


\section{The purpose of the study}\label{sec:the-purpose-of-the-study}
\justifying
Focusing on the effective ways that Ai can enhance learning experiences within humanities.
Hence, the study aims to
explore the impact of Ai tools on students of humanities' productivity and performance.
% 3rd


\section{The Rationale and significance of the study}\label{sec:the-rationale-and-significance-of-the-study}
\justifying
The epidemic accessibility and abundance of Ai shows that 73\% of US companies have already
implemented Ai in some businesses, according to\textcite{pricewaterhousecoopers}.
Hence, the fame of using Ai in the last years prompted researchers to explore effective ways of utilizing Ai tools
for enhancing humans' productivity, including education.
This research paper tackles the AI-driven tools in a such framework that
deals with the problem of harnessing it effectively for enhancing learning experiences in the humanities.
Overall, this study will provide a more in-depth and detailed understanding of the use of Ai tools within humanities.
\section{Research questions and hypotheses}\label{sec:research-questions-and-hypotheses}

\subsection{Research questions}\label{subsec:research-questions}
\justifying
The study seeks to investigate the potential ways
of harnessing Artificial Intelligence for
Enhanced Learning Experiences in the Humanities.
Hence,
the following research questions will be addressed in this paper:
\begin{itemize}
    \item What are the most effective ways to utilize AI-driven
    tools for enhancing learning experiences in higher education, 
    especially in the humanities?
    \item What is the impact of AI-driven pedagogical tools 
    on university students' academic performance 
    and engagement in the humanities?
    \item What are the challenges and opportunities associated 
    with using AI in higher education in Morocco, 
    specifically in the humanities?
\end{itemize}
\subsection{Hypotheses}\label{subsec:hypotheses}
\justifying
Following intended objectives, these hypotheses have been developed:
\begin{itemize}
    \item Students who use AI-driven tools reveal better learning outcomes
    compared to those who do not in higher education, specifically in the humanities.
    \item AI-driven tools are significantly improving academic
    performance and engagement in the humanities.
    \item There are challenges and opportunities are associated with using Ai in higher
    education in Morocco, specifically in the humanities.
\end{itemize}

% 5th


\section{The Organization of the paper}\label{sec:the-organization-of-the-paper}
\justifying
The monograph comprises five chapters with each chapter contributing to the study.
To break it down:
the first chapter servers as an overview of the study.
It includes the study problem\ref{sec:problem-statement}, the purpose of the study\ref{sec:the-purpose-of-the-study},
its rational and significance\ref{sec:the-rationale-and-significance-of-the-study},
and the study's questions and hypotheses\ref{sec:research-questions-and-hypotheses}.
The second chapter is a review of literature.
It focuses on studies conducted on the use of AI in education
to showcase current trends, challenges, and effective strategies
for utilizing AI-driven tools.
The third chapter is made up to provide a comprehensive explanation of data-collection.
It describes the research design, participants, instrument, and relevant procedures
adopted for analysis.
As progressed to the findings chapter will analysis, interpret, and discuss data-collection in depth.
Additionally, the chapter aims to either validate or reject the hypotheses of the study.
Lastly, in the concluding chapter, the attention will be directed
towards summarizing research objectives, methodology, employed
and key findings.
Furthermore, this section will address the limitations and
implications of the study while also providing suggestions
for further studies.

% new txt
%The second chapter is a review of literature.
%It reviews the most existing studies on Ai in education to highlight current
%trends, challenges, and potential strategies for utilizing AI-driven tools.
%The third chapter is designed to provide a comprehensive explanation of data-collection.
%It describes the research design, participants, instrument, and relevant procedures
%adopted for analysis.
%The finding chapter will analysis, interpret, and discuss data-collection in depth.
%Additionally, the chapter aims to either validate or reject the hypotheses of the study.
%Finally, the concluding chapter will focus on a summary of research objectives,
%methodology, and findings.
%In addition, it will contain the shortcomings and implications of the study,
%as well as suggestions for further studies.


