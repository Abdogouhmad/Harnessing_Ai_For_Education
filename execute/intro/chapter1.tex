\chapter{INTRODUCTORY CHAPTER}
%\addcontentsline{toc}{chapter}{INTRODUCTORY CHAPTER}
\section{Problem statement}
\justifying
Artificial Intelligence is revolutionized , and its potential
to transform higher education, is increasingly appreciated. However, despite the emergency of artificial intelligence and
Ai-driven tools like \say{ChatGPT} which has a significant capabilities,
there remains a gap in comprehending how effectively Ai can
enhance learning experiences within humanities. While \say{ChatGPT}
and similar Ai-driven tools have gained prominence across various
industries since its release in late November 2022, it has not been
fully utilized in the field of education. Rather than serving users
to accomplish their tasks for better productivity. As a result, questions
has been raised about the results in fostering genuine education engagement
and knowledge acquiring among student with the existence of Ai-driven tools.
Therefore, the central problem of this study is How can Ai be harnessed
in education to enhance learning experiences in the humanities.
% 2nd section
\section{The purpose of the study}
\justifying
% This study is an attempt to investigate the potential ways that artificial intelligence
% (Ai) can enhancing learning experiences within the humanities.
% Particularly, the study seeks to uncover the effective ways for integrating
% Ai or Ai-driven tools into the field of education for better productivity.
The study is an attempt to examine and investigate the potential ways to integrate Ai in higher education in Morocco.
Focusing on the effective ways that Ai can enhance learning experiences within humanities. Hence, the study aims to
explore the impact of Ai tools on students of humanities' productivity and performance.
% 3rd
\section{Rationale and significance of the study}
\justifying
The epidemic accessibility and abundance of Ai shows that 73\% of US companies have already
implemented Ai in some businesses according to \textcite{pricewaterhousecoopers} .
Hence, the fame of using Ai in the last years prompted researchers to explore effective ways of utilizing Ai tools
for enhancing humans' productivity including education. This research paper tackles the Ai-driven tools in a such framework that
deals with the problem of harnessing it effectively for enhancing learning experiences in the humanities.
Overall, this study will provide a more in-depth and detailed understanding of the use of Ai tools within humanities.
\section{Research questions and hypotheses}
\subsection{Research questions}
\justifying
\begin{itemize}
    \item question 1
    \item question 2
    \item question 3
\end{itemize}
\subsection{Hypotheses}
\justifying
\begin{itemize}
    \item hypotheses 1
    \item hypotheses 2
    \item hypotheses 3
\end{itemize}

% 5th

\section{Organization of the study}
\justifying
\lipsum[1]


