\chapter{INTRODUCTORY CHAPTER}
\section{Problem statement}
\justifying
Artificial Intelligence is revolutionized , and its potential
to transform higher education specifically in the humanities, is increasingly
appreciated. However, despite the emergency of artificial intelligence and
Ai-driven tools like ChatGPT which has a significant capabilities,
there remains a gap in comprehending how effectively harness Ai into
education to enhance learning experiences in humanities. While ChatGPT
and similar Ai-driven tools have gained prominence across various
industries since its release in late November 2022, it has not been
fully utilized in the field of education. Rather than serving users
to accomplish their tasks or even learn from it. As a result, questions
has been raised about the results in fostering genuine education engagement
and knowledge acquiring among student with the existence of Ai-driven tools. Therefore, this study aims
the central problem of this research paper is How can Ai be harnessed
in education to enhance learning experiences in the humanities?
% 2nd section
\section{The purpose of the study}
\justifying
% This study is an attempt to investigate the potential ways that artificial intelligence
% (Ai) can enhancing learning experiences within the humanities.
% Particularly, the study seeks to uncover the effective ways for integrating
% Ai or Ai-driven tools into the field of education for better productivity.
The study is an attempt to examine and investigate the potential ways to integrate Ai in higher education in Morocco.
Focusing on the effective ways that Ai can enhance learning experiences within humanities. Hence, the study aims to 
explore the impact of Ai tools on students of humanities' productivity and performance.
% 3rd
\section{Rationale and significance of the study}
\justifying
\lipsum[1]
% 4th
\section{Research questions and hypotheses}
\subsection{Research questions}
\justifying
\begin{itemize}
    \item question 1
    \item question 2
    \item question 3
\end{itemize}
\subsection{Hypotheses}
\justifying
\begin{itemize}
    \item hypotheses 1
    \item hypotheses 2
    \item hypotheses 3
\end{itemize}

% 5th

\section{Organization of the study}
\justifying
\lipsum[1]


